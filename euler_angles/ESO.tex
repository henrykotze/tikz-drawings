\documentclass[crop, tikz]{standalone}
\usepackage{tikz}
\usepackage{tikz-3dplot}
\usepackage{pgfplots}
% Workaround for making use of externalization possible
% -> remove hardcoded pdflatex and replace by lualatex
\usepgfplotslibrary{external}

% Redefine rotation sequence for tikz3d-plot to z-y-x
\newcommand{\tdseteulerxyz}{
	\renewcommand{\tdplotcalctransformrotmain}{%
		%perform some trig for the Euler transformation
		\tdplotsinandcos{\sinalpha}{\cosalpha}{\tdplotalpha}
		\tdplotsinandcos{\sinbeta}{\cosbeta}{\tdplotbeta}
		\tdplotsinandcos{\singamma}{\cosgamma}{\tdplotgamma}
		%
		\tdplotmult{\sasb}{\sinalpha}{\sinbeta}
		\tdplotmult{\sasg}{\sinalpha}{\singamma}
		\tdplotmult{\sasbsg}{\sasb}{\singamma}
		%
		\tdplotmult{\sacb}{\sinalpha}{\cosbeta}
		\tdplotmult{\sacg}{\sinalpha}{\cosgamma}
		\tdplotmult{\sasbcg}{\sasb}{\cosgamma}
		%
		\tdplotmult{\casb}{\cosalpha}{\sinbeta}
		\tdplotmult{\cacb}{\cosalpha}{\cosbeta}
		\tdplotmult{\cacg}{\cosalpha}{\cosgamma}
		\tdplotmult{\casg}{\cosalpha}{\singamma}
		%
		\tdplotmult{\cbsg}{\cosbeta}{\singamma}
		\tdplotmult{\cbcg}{\cosbeta}{\cosgamma}
		%
		\tdplotmult{\casbsg}{\casb}{\singamma}
		\tdplotmult{\casbcg}{\casb}{\cosgamma}
		%
		%determine rotation matrix elements for Euler transformation
		\pgfmathsetmacro{\raaeul}{\cacb}
		\pgfmathsetmacro{\rabeul}{\casbsg - \sacg}
		\pgfmathsetmacro{\raceul}{\sasg + \casbcg}
		\pgfmathsetmacro{\rbaeul}{\sacb}
		\pgfmathsetmacro{\rbbeul}{\sasbsg + \cacg}
		\pgfmathsetmacro{\rbceul}{\sasbcg - \casg}
		\pgfmathsetmacro{\rcaeul}{-\sinbeta}
		\pgfmathsetmacro{\rcbeul}{\cbsg}
		\pgfmathsetmacro{\rcceul}{\cbcg}
	}
}

% Set the plot display orientation
% Syntax: \tdplotsetdisplay{\theta_d}{\phi_d}
\tdplotsetmaincoords{60}{140}

\pgfmathsetmacro{\zRot}{10}
\pgfmathsetmacro{\yRot}{10}
\pgfmathsetmacro{\xRot}{10}
%%%%%%%%% Using standard euler angles implemented in default tikz-3dplot implementation
\begin{document}
	%%%%%%%%%%%%% Z-Y-Z
	\begin{tikzpicture}[scale=2.5,tdplot_main_coords]
	
	% Set origin of main (body) coordinate system
	\coordinate (O) at (0,0,0);
	
	% Draw main coordinate system
	\draw[red, ,->] (0,0,0) -- (1,0,0) node[anchor=north east]{$x_{\mathcal{I}}$};
	\draw[red, ,->] (0,0,0) -- (0,1,0) node[anchor=north west]{$y_{\mathcal{I}}$};
	\draw[red, ,->] (0,0,0) -- (0,0,1) node[anchor=south]{$z_{\mathcal{I}}$};
	
	% Rotate to final frame
	\tdplotsetrotatedcoords{\zRot}{\yRot}{\xRot}
	\draw[thick,tdplot_rotated_coords,->, cyan] (0,0,0) -- (1,0,0) node[anchor=west]{$x_{\mathcal{B}}$};
	\draw[thick,tdplot_rotated_coords,->, cyan] (0,0,0) -- (0,1,0) node[anchor=west]{$y_{\mathcal{B}}$};
	\draw[thick,tdplot_rotated_coords,->, cyan] (0,0,0) -- (0,0,1) node[anchor=south]{$z_{\mathcal{B}}$};
	
	%Draws circle representing the rotated planes. Each of these should be "pointed" by two arrows.
	\tdplotdrawarc[dashed,tdplot_rotated_coords,->,color=black]{(0,0,0)}{1}{0}{350}{anchor=south west,color=black}{}
	\tdplotsetrotatedthetaplanecoords{0}
	\tdplotdrawarc[dashed,tdplot_rotated_coords,->,color=blue]{(0,0,0)}{1}{0}{350}{anchor=south west,color=blue}{}
	\tdplotsetrotatedthetaplanecoords{90}
	\tdplotdrawarc[dashed,tdplot_rotated_coords,->,color=red]{(0,0,0)}{1}{0}{350}{anchor=south west,color=red}{}
	\end{tikzpicture}
	
	%%%%%% Change the rotation matrix in order to use Tait-Bryan angles
	\tdseteulerxyz
	%%%%%%%%%%%%% Z-Y-X

\end{document}